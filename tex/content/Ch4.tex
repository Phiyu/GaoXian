\chapter{最小作用量原理}

\problem{选取合适的广义坐标,求习题\ref{problem2-2}中系统的拉格朗日量和运动方程。}

\begin{solution}
	\begin{figure}[h]
		\centering
		\includegraphics[width=0.3\textwidth]{content/Figures/2-2}
		\caption{ }
		\label{fig:4-1}
	\end{figure}
    原题如图\ref{fig:4-1}所示,选取\(m_2\)与小孔的距离\(r\)和\(m_2\)转动的角度\(\theta\)作为广义坐标。记绳长为\(l\),则\(m_1,m_2\)与小孔的距离分别可以表示为
    \begin{align*}
    	\begin{cases}
    		r_1=l-r\\
    		r_2=r
    	\end{cases}
    \end{align*}
    系统的动能可表示为
    \[T=\frac{1}{2}m_1 \dot{r}^2+\frac{1}{2}m_2(\dot{r}^2+r^2 \dot{\theta}^2)=\frac{1}{2}(m_1+m_2)\dot{r}^2+\frac{1}{2}m_2 r^2\dot{\theta}^2\]
    而势能则可计算为
    \[V=-m_1 g(l-r)\]
    于是Lagrangian为
    \[L=T-V=\frac{1}{2}(m_1+m_2)\dot{r}^2+\frac{1}{2}m_2 r^2\dot{\theta}^2+m_1 g(l-r)\]
    代入拉格朗日方程
    \[\frac{\partial L}{\partial r}-\frac{\dd }{\dd t}\frac{\partial L}{\partial \dot{r}}=0,\frac{\partial L}{\partial \theta}-\frac{\dd }{\dd t}\frac{\partial L}{\partial \dot{\theta}}=0\]
    得到运动方程为
    \begin{align*}
    	\begin{cases}
    		(m_1+m_2)\ddot{r}=m_2 r\dot{\theta}^2-m_1 g\\
    		\frac{\dd}{\dd t}(m_2 r^2 \dot{\theta})=0
    	\end{cases}
    \end{align*}
\end{solution}

\problem{选取合适的广义坐标,求习题\ref{problem2-3}中系统的拉格朗日量和运动方程。}
\begin{solution}
	\begin{figure}[h]
		\centering
		\includegraphics[width=0.3\textwidth]{content/Figures/2-3}
		\caption{ }
		\label{fig:4-2}
	\end{figure}
	原题如图\ref{fig:4-2}所示,无约束时的广义坐标为$\{x, r_1, r_2\}$,其中 $x$ 为楔块在水平面上的位置,$r_1$ 和 $r_2$ 分别是两个粒子到楔块尖端的距离,约束方程为
	\[\phi (r_1, r_2) = r_1 + r_2 - l = 0\]
	于是选取广义坐标\(r,x\),而\(r_1,r_2\)分别为
    \begin{align*}
		\begin{cases}
			r_1=r\\
			r_2=l-r
		\end{cases}
	\end{align*}
	为了方便,选取坐标原点到楔块最上方顶点的距离为\(x\),且两点等高,那么对两个块分别有
    \begin{align*}
		\begin{cases}
			x_1=x-r\cos{\theta_1},y_1=-r \sin{\theta_1}\\
			x_2=x+(l-r)\cos{\theta_2},y_2=-(l-r)\sin{\theta_2}
		\end{cases}
	\end{align*}
	系统的动能为
	\begin{align*}
		T&=\frac{1}{2}M\dot{x}^2 +\frac{1}{2}m_1(\dot x_1^2+\dot y_1^2)+\frac{1}{2}m_2(\dot x_2^2+\dot y_2^2)\\
		&=\frac{1}{2}M\dot{x}^2+\frac{1}{2} m_1 \left(\sin^2{\theta_1} \dot r^2+\left(\dot x-\cos{\theta_1} \dot r\right)^2\right)+\frac{1}{2} m_2 \left(\sin^2{\theta_2} \dot r^2+\left(\dot x-\cos{\theta_2} \dot r\right)^2\right)\\
		&=\frac{1}{2} \left((M+m_1+m_2) \dot x^2-2 (m_1 \cos{\theta_1}+m_2 \cos{\theta_2})\dot r \dot x +(m_1+m_2) \dot r^2\right)\\
		&=\frac{1}{2}(M+m_1+m_2) \dot x^2-(m_1 \cos{\theta_1}+m_2 \cos{\theta_2})\dot r \dot x +\frac{1}{2}(m_1+m_2) \dot r^2
	\end{align*}
	势能则为
	\[V=m_1 g y_1+m_2 g y_2=-m_1 g r\cos{\theta_1}-m_2 g(l-r)\cos{\theta_2}\]
	拉格朗日量则是
	\[L=T-V=\frac{1}{2}(M+m_1+m_2) \dot x^2-(m_1 \cos{\theta_1}+m_2 \cos{\theta_2})\dot r \dot x +\frac{1}{2}(m_1+m_2) \dot r^2+(m_1\cos{\theta_1}-m_2\cos{\theta_2})g r+m_2 g l \cos{\theta_2}\]
	代入拉格朗日方程
	\begin{align*}
		\frac{\partial L}{\partial x}-\frac{\dd}{\dd t}\frac{\partial L}{\partial \dot{x}}=0\\
		\frac{\partial L}{\partial r}-\frac{\dd}{\dd t}\frac{\partial L}{\partial \dot{r}}=0
	\end{align*}
	计算得到运动方程为
	\begin{align*}
		(m_1\cos{\theta_1}+m_2\cos{\theta_2}) \ddot{r}-(M+m_1+m_2)\ddot{x}=0\\
		-(m_1+m_2)\ddot{r}+(m_1\cos{\theta_1}+m_2\cos{\theta_2})\ddot{x}+(m_1\cos{\theta_1}-m_2\cos{\theta_2})g=0
	\end{align*}
\end{solution}


\problem{选取合适的广义坐标,求习题\ref{problem2-4}中系统的拉格朗日量和运动方程。}
\begin{solution}
    \begin{figure}[h]
    	\centering
    	\includegraphics[width=0.2\textwidth]{content/Figures/2-4}
    	\caption{ }
    	\label{fig:4-3}
    \end{figure}
    原题如\ref{fig:4-3}所示,由于只有一个自由度,不妨选取\(x\)为广义坐标。在柱坐标系中,粒子坐标为
    \begin{align*}
    	\rho&=x\\
    	\theta&=\omega t\\
    	z&=z(x)
    \end{align*}
    其动能表达式为
    \[T=\frac{1}{2}m(\dot{\rho}^2+\rho^2 \dot{\theta}^2+\dot{z}^2)=\frac{1}{2}m (1+z'(x)^2)\dot{x}^2+\frac{1}{2}m\omega^2 x^2\]
    势能则为
    \[V=mgz=mg z(x)\]
    拉格朗日量为
    \[L=T-V=\frac{1}{2}m (1+z'(x)^2)\dot{x}^2+\frac{1}{2}m\omega^2 x^2-mg z(x)\]
    代入欧拉-拉格朗日方程
    \[\frac{\partial L}{\partial x}-\frac{\dd}{\dd t}\frac{\partial L}{\partial \dot{x}}=0\]
    得到运动方程为
    \[m\omega^2 x-mgz'(x)-\frac{\dd}{\dd t}\left[m(1+z'(x)^2)\dot{x}\right]=0\]
\end{solution}


\problem{}
\begin{solution}
    待施工.
\end{solution}


\problem{}
\begin{solution}
    待施工.
\end{solution}


\problem{}
\begin{solution}
    待施工.
\end{solution}


\problem{}
\begin{solution}
    待施工.
\end{solution}


\problem{}
\begin{solution}
    待施工.
\end{solution}



\problem{}
\begin{solution}
    待施工.
\end{solution}



\problem{}
\begin{solution}
    待施工.
\end{solution}



\problem{考虑与标量场相互作用的粒子作用量的 4 维形式和 3 维形式, 分别求粒子运动方程的 4 维形式和 3 维形式.}
\begin{solution}
    Minkowski 时空标量场中的粒子,其作用量为
    \[
        S = - m c \int \dd \tau \,\me^{\Phi} \sqrt{- \eta_{\mu \nu} \frac{\dd x^\mu}{\dd \tau} \frac{\dd x^\nu}{\dd \tau}},
    \]
    将被固有时 $\tau$ 参数化后的世界线 (作用量) 对 $x^\mu$ 作变分:
    \begin{align*}
        \delta S &= - mc \int \dd \tau \left[\frac{\pp}{\pp x^\mu} \left(\me^{\Phi} \sqrt{- \eta_{\mu \nu} \frac{\dd x^\mu}{\dd \tau} \frac{\dd x^\nu}{\dd \tau}}\right) \delta x^\mu + \frac{\pp}{\pp \left(\frac{\dd x^\mu}{\dd \tau}\right)} \left(\me^{\Phi} \sqrt{- \eta_{\mu \nu} \frac{\dd x^\mu}{\dd \tau} \frac{\dd x^\nu}{\dd \tau}}\right)\delta \left(\frac{\dd x^\mu}{\dd \tau}\right)\right]  \\
        &= - mc \int \dd \tau \left[\frac{\pp \Phi}{\pp x^\mu} \me^{\Phi} \sqrt{- \eta_{\mu \nu} \frac{\dd x^\mu}{\dd \tau} \frac{\dd x^\nu}{\dd \tau}} \delta x^\mu - \me^{\Phi} \frac{\eta_{\mu \nu} \frac{\dd x^\nu}{\dd \tau}}{\sqrt{- \eta_{\mu \nu} \frac{\dd x^\mu}{\dd \tau} \frac{\dd x^\nu}{\dd \tau}}} \delta \left(\frac{\dd x^\mu}{\dd \tau}\right)\right] \\
        & \simeq -mc \int \dd \tau \left[\frac{\pp \Phi}{\pp x^\mu} \me^{\Phi} \sqrt{- \eta_{\mu \nu} \frac{\dd x^\mu}{\dd \tau} \frac{\dd x^\nu}{\dd \tau}} + \frac{\dd}{\dd \tau} \left(\me^{\Phi} \frac{\eta_{\mu \nu} \frac{\dd x^\nu}{\dd \tau}}{\sqrt{- \eta_{\mu \nu} \frac{\dd x^\mu}{\dd \tau} \frac{\dd x^\nu}{\dd \tau}}}\right) \right] \delta x^\mu,
    \end{align*}
    因为 4-速度的模方是常数 $\eta_{\mu \nu} \frac{\dd x^\mu}{\dd \tau} \frac{\dd x^\nu}{\dd \tau} = - c^2$, 且由度规升降 $\eta_{\mu \nu} \frac{\dd x^\nu}{\dd \tau} \equiv \frac{\dd x_\mu}{\dd \tau}$, 所以我们可以写出 Euler-Lagrange 方程:
    \begin{align*}
        - \frac{1}{mc} \frac{\delta S}{\delta x^\mu} &= \frac{\pp \Phi}{\pp x^\mu} \me^\Phi c + \frac{\dd}{\dd \tau} \left(\frac{\me^\Phi}{c} \frac{\dd x_\mu}{\dd \tau}\right) = 0 \\
        &= c \frac{\pp \Phi}{\pp x^\mu} \me^\Phi + \frac{1}{c} \frac{\dd x_\mu}{\dd \tau} \frac{\pp \me^\Phi}{\pp \tau} + \frac{\me^\Phi}{c} \frac{\dd^2 x_\mu}{\dd \tau^2} = 0 \\
        &= c \frac{\pp \Phi}{\pp x^\mu} \me^\Phi + \frac{1}{c} \frac{\dd x_\mu}{\dd \tau} \frac{\pp x^\nu}{\pp \tau} \frac{\pp \Phi (x^\mu)}{\pp x^\nu} \me^\Phi + \frac{\me^\Phi}{c} \frac{\dd^2 x_\mu}{\dd \tau^2} = 0,
    \end{align*}
    即 4 维形式的运动方程
    \[
        \frac{\dd^2 x_\mu}{\dd \tau^2} + \frac{\pp \Phi}{\pp x^\nu} \frac{\pp x^\nu}{\pp \tau} \frac{\dd x_\mu}{\dd \tau} + c^2 \frac{\pp \Phi}{\pp x^\mu} = 0, \quad \mu = 0, 1, 2, 3.
    \]
    标量场作用下的作用量的 3 维形式为
    \[
        S = - m c \int \dd t\,\me^\Phi \sqrt{1- \frac{\delta_{ij}}{c^2} \frac{\dd x^i}{\dd t} \frac{\dd x^j}{\dd t}},
    \]
    对 3 维坐标 $x^i$ 作变分:
    \begin{align*}
        \delta S &= - mc \int \dd t \left[\frac{\pp}{\pp x^i} \left(\me^\Phi \sqrt{1- \frac{\delta_{ij}}{c^2} \frac{\dd x^i}{\dd t} \frac{\dd x^j}{\dd t}}\right) \delta x^i + \frac{\pp}{\pp \left(\frac{\dd x^i}{\dd t}\right)} \left(\me^\Phi \sqrt{1- \frac{\delta_{ij}}{c^2} \frac{\dd x^i}{\dd t} \frac{\dd x^j}{\dd t}}\right) \delta \left(\frac{\dd x^i}{\dd t}\right)\right] \\
        &= - mc \int \dd t \left[\frac{\pp \Phi}{\pp x^i} \me^\Phi \sqrt{1- \frac{\delta_{ij}}{c^2} \frac{\dd x^i}{\dd t} \frac{\dd x^j}{\dd t}} \delta x^i - \me^\Phi \frac{\delta_{ij} \frac{\dd x^j}{\dd t} \delta \left(\frac{\dd x^i}{\dd t}\right)}{c^2 \sqrt{1- \frac{\delta_{ij}}{c^2} \frac{\dd x^i}{\dd t} \frac{\dd x^j}{\dd t}}}\right] \\
        & \simeq - mc \int \dd t \left[\frac{\pp \Phi}{\pp x^i} e^\Phi \sqrt{1 - \frac{\vec{v}^2}{c^2}} + \frac{\dd}{\dd t} \left(\frac{\me^\Phi}{c^2} \frac{\frac{\dd x_i}{\dd t}}{\sqrt{1 - \frac{\vec{v}^2}{c^2}}}\right)\right] \delta  x^i, \\
        - \frac{\delta S}{\delta x^i} &= m c \frac{\pp \Phi}{\pp x^i} \me^\Phi \sqrt{1 - \frac{\vec{v}^2}{c^2}} + \frac{\me^\Phi}{c^2} \dot{\Phi} \frac{mc \frac{\dd x_i}{\dd t}}{\sqrt{1 - \frac{\vec{v}^2}{c^2}}} + \frac{\me^\Phi}{c^2} \frac{mc \frac{\dd^2 x_i}{\dd t^2}}{\sqrt{1 - \frac{\vec{v}^2}{c^2}}} = 0,
    \end{align*}
    3-动量定义为 $p_i \equiv m \frac{\dd x_i}{\dd t} \frac{\dd t}{\dd \tau} \equiv m \frac{\dd x_i}{\dd t} \frac{1}{\sqrt{1 - \frac{\vec{v}^2}{c^2}}}$, 则上式整理为, 
    \[
        \frac{\me^\Phi}{c} \left(\dot{\Phi} p_i + \dot{p}_i\right) + mc \frac{\pp \Phi}{\pp x^i} \me^\Phi \sqrt{1 - \frac{\vec{v}^2}{c^2}} = 0,
    \]
    即 3 维形式的运动方程
    \[
        \dot{p}_i + \dot{\Phi} p_i + mc^2 \sqrt{1 - \frac{\vec{v}^2}{c^2}} \frac{\pp \Phi}{\pp x^i} = 0, \quad i = 1, 2, 3.
    \]
\end{solution}



\problem{电磁场中带电粒子作用量的 4 维形式和 3 维形式分别为
    \begin{align*}
        S = \int \dd \tau L, \quad L = - m c \sqrt{- u_\mu u^\mu} + \frac{e}{c} A_\mu u^\mu, \\
        S = \int \dd \tau L, \quad L = - m c^2 \sqrt{1 - \frac{\vec{v}^2}{c^2}} - e \Phi + \frac{e}{c} \vec{v} \cdot \vec{A}.
    \end{align*}
    \begin{enumerate}[label=(\arabic*)]
        \item 求粒子的 4-共轭动量 $P_\mu \equiv \frac{\pp L}{\pp u^\mu}$ 和 3-共轭动量 $P_i \equiv \frac{\pp L}{\pp \dot{x}^i}$;
        \item 分别求粒子运动方程的 4 维形式和 3 维形式; 
        \item 若 $E$ 由式
        \[
            E := c p^0 = m c u^0 = m c^2 \frac{\dd t}{\dd \tau} = \frac{m c^2}{\sqrt{1 - \frac{\vec{v}^2}{c^2}}}
        \]
        给出, 证明 $\frac{\dd E}{\dd t} = e \vec{v} \cdot \vec{E}$.
    \end{enumerate}
}

\begin{solution}
    \begin{enumerate}[label=(\arabic*)]
        \item 待施工.
    \end{enumerate}
\end{solution}
